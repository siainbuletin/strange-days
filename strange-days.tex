%%%%%%%%%%%%%%%% PREAMBLE %%%%%%%%%%%%%%%%%%%%
\documentclass[12pt]{article}

%% LANGUAGE SETTINGS %%
\usepackage{polyglossia}
\setmainlanguage{romanian}
\setotherlanguage[variant=ancient]{greek}

\usepackage[de]{luaquotes}

%% FONT SETTINGS %%
\usepackage{fontspec}
\usepackage{trace}

\setmainfont{et-book-roman-old-style-figures.tff}

\newfontfamily[Script=Greek]{Symbol}

\usepackage{setspace}
\onehalfspacing
\setlength{\parindent}{25pt}
\defaultfontfeatures{Scale=MatchLowercase}

%% QUOTATION %%
\usepackage[
strict=true,
autopunct=true]
{csquotes}

%% CROSS-REFERENCING %%
\usepackage{hyperref}

%% BIBLIOGRAPHY SETUP %%
\usepackage[
	backend=biber, 
	sorting=none,
	style=numeric,
	autocite=plain,
]{biblatex}
\addbibresource{strange-days.bib}

%% METADATA
\title{Strange Days: problematizarea scoptofiliei}
\author{Sia Piperea}

%%%%%%%%%%%%%%%%%%%%%%%%%%%%%%%%%%%%%%%%%%%%%

\begin{document}
	 	
	\maketitle
	
	\begin{description}
		\item[scoptofilie] gr. \begin{greek}[variant=ancient] σκοπός \end{greek} (skopós, \dedouble privitor\sqtworight) +‎ \begin{greek}[variant=ancient] φιλία \end{greek} (philía, \dedouble iubire\sqtworight)\break
		 (patologie) Dependența (sexuală) de a observa deschis acte sexuale fără jenă (spre deosebire de voyeurism)
	\end{description}
			 	
	Cyberpunk a început ca o mișcare de renaștere science fiction, în mare parte datorită romanului \textit{Neuromancer} (William Gibson).\footnote{Există zvonuri cum că ori Bigelow \cite{Salza1994a}, ori scenaristul și producătorul filmului, James Cameron \cite{Henry2023a}, ar fi trebuit să adapteze povestirea \textit{Burning Chrome} a lui Gibson, însă proiectul nu s-a putut materializa (în mare parte deoarece nu era pe placul lui Gibson). Ce a rămas din acea adaptare s-a transformat în cele din urmă în \textit{Strange Days}.} Spre deosebire de utopianismul caracteristic SF-urilor precendente, cyberpunk pune accentul pe conflictul dintre om și mașinărie, o criză a identității și o stare de anxietate care doar s-a complicat odată cu trecerea deceniilor.\par
	
	Date fiind popularitatea genului cyberpunk și cariera înfloritoare de care Kathryn Bigelow se bucura la începutul anilor '90, \textit{Strange Days} ar fi trebuit să fie un succes comercial. În realitate, nu a reușit să strângă nici măcar un sfert din bugetul alocat. Recenziile contemporane menționează proasta strategie de marketing drept justificare pentru eșecul filmului \cite{McCarthy1995a}. Mulți alții critică abordarea regizoarei, considerând că discursul său politic este peticit și forțat asupra poveștii.\footnote{Ce-i drept, Bigelow a menționat că faptul că a urmărit revoltele din 1992 a motivat-o și mai tare să demareze producția pentru \textit{Strange Days}.} Astfel de obiecții sunt probabil nesurprinzătoare, dat fiind că procesul lui O.J. Simpson bântuie cronicile toamnei 1995\footnote{Simpson fusese achitat la doar 10 zile înainte de premiera filmului.} la fel de mult cât acesta o bântuia pe regizoare în timpul producției \cite{Willistein1995a}.\par
	
	Contrar \dedouble tradiției\sqtworight\:cyberpunk de a se refugia în speculație, camuflându-și astfel comentariul socio-politic, Bigelow este directă. Ea evită fantezismul tipic SF-ului și te subjugă unghiului subiectiv într-un efort de a crea înțelegere și empatie \cite{Ebert1995a}. Însă audiența tipică filmelor de gen nu este atât de tolerantă la coerciție. Prin contrast, chiar dacă în prezent \textit{The Matrix} (1999, r. Lana și Lilly Wachowski) este acceptat ca fiind o alegorie clară pentru experiența transgender, când și-a avut premiera a fost digerat mai ușor datorită posibilității de ocolire a problemelor filosofice pe care le ridică.\par 
	
	Unii critici au menționat că dacă \textit{Strange Days} ar fi fost regizat de un bărbat, acesta ar fi fost etichetat ca misogin \cite{MirasolBigelowUncanny2010}. Visceralitatea imaginilor din \textit{Strange Days} a fost considerată gratuită, lipsită de un ochi moralizator, fiind un simplu experiment și o joacă de-a film \textit{snuff} \cite{Guthmann1995a,DenbyPeopleStrange1995}.\footnote{Denby a numit scena violului „the sickest sequence in modern movies”.}\par
	
	Critici similare au continuat de-a lungul carierei ei, fiind acuzată aproape 20 de ani mai târziu de propagandă CIA în Zero Dark Thirty \cite{Vishnevetsky2012a}. Cumva, cu \textit{Strange Days} a reușit să alieneze toate grupările separate care-i compuneau audiența țintă. Deoarece Bigelow este devotată reprezentării fidele, îmbrățișând grotescul ca o consecință naturală a observației libere, filmul ar putea fi recunoscut mai curând pentru sensibilitățile sale de film de artă, însă finalul tipic pe care-l adaugă scoate în evidență limitele genului în care Bigelow încearcă să se joace, considerându-l un mediu propice transgresiunii.\par
	
	Bigelow a fost lăudată pentru măiestria tehnică de care a dat dovadă prin utilizarea subiectivului, încercând să întreacă limitele mediului filmic în scopul simulării tehnologiei \dedouble SQUID\sqtworight. Realitatea virtuală este portretizată ca un asalt asupra simțurilor, astfel, fără visceralitate, fără violență, tema principală nu ar fi putut fi transmisă. Limbajul ei filmic îi permite evitarea CGI-ului; în schimb, voyeurismul devine conceptul de punte care ne apropie temporal de Los Angeles-ul creat de Bigelow \cite{Hultkrans2010a}.\par
	
	 Regizoarea înțelege că forța cinema-ului stă în crearea imersiunii, creând o senzație de \textit{aproape} de care cuvântul scris nu ar fi niciodată capabil, însă această magie este spulberată odată ce devii conștient de spațiul dintre tine și imaginea proiectată. Venind dinspre artele plastice abstracte, Bigelow s-a reorientat în anii '80 către film, simțind nevoia de a comunica idei într-o paradigmă artistică mai puțin abstractă însă poate mai puternică la nivel emoțional. \textit{Strange Days} este un film ce \textit{tânjește}, încercând să imagineze imposibilul: venirea următorului mileniu.
	 
	 Astfel, de unde se poziționa în acel moment, filmul reprezenta o evoluție în posibilități de exprimare.\par
	 
	\begin{displaycquote}{Hultkrans2010a}
		Filmul mi se prezenta ca această unealtă socială incredibilă care nu-ți cerea nimic mai mult decât să-i acorzi atenție timp de douăzeci de minute sau două ore.
	\end{displaycquote}
	
	În viziunea regizoarei, această versiune a Los Angelesului este rezultatul unei dezvoltări naturale de pe urma revoltelor anului 1992. Orașul este puternic militarizat, într-o încercare a poliției de a relua control asupra populației. Această decizie se reflectă mai ales asupra celor marginali, în special comunității Afro-Americane. Acest LA nu este departe în viitorul său; punând accent pe ultimele zile ale mileniului, este o exprimare a anxietăților asupra a ceea ce urmează să fie continuat și în anul 2000.\par
	
	În ultimii 30 de ani, realitatea virtuală și-a pierdut din caracterul fascinant. Căștile de VR sunt greoaie, caraghioase, iar experiențele create sunt simulări imperfecte și încorsetate. SQUID este și el limitat, însă doar de amintirile pe care le prezintă.\par
	
	Dispozitivul este conceput ca o simplă bucată de gel echipată cu conectori neuronali ce facilitează accesul nemărginit la adâncimile cortexului prefrontal, dar se bucură de mistificarea pe care celuloidul a pierdut-o pe parcursul secolului XX. Oamenii nu se mai sperie de filmarea fraților Lumiere a trenului filmat din față \cite{Willistein1995a}, deoarece peretele dintre spectator și celuloid este acum opac. Legislativul prezent în LA nu este capabil să-l controleze; SQUID se formează asemeni unui Leviathan, mai presus de suma părților sale componente.\par
	
	Amintirile și informația transmisă devine un scop în sine: cu cât sunt produse mai multe, cu atât mai bine. Oamenii devin date \cite{Vishnevetsky2012a}; simple fragmente de amintire comodificate, puse în loop și incapacitat de facultățile decizionale. Moartea inocenților este doar un prilej de plăcere pentru amatorii de senzații tari. Bigelow ni-l arată pe Lenny (Ralph Fiennes) plătind actori pentru înregistrări pentru a-și maximiza profitul. Utilizatorul SQUID devine un simplu corp gol, așteptând, \textit{cerând} să fie ocupat de către persoana dispusă să plătească prețul cerut. Persoana înregistrată este și ea un simplu instrument pentru crearea amintirilor care \textit{merită} să fie (re)trăite în detrimentul vieții reale.\par
	
	Lenny își dorește să se retragă din lume deoarece suita de amintiri favorite cu Faith (Juliette Lewis) este singurul lucru care-l mai poate face fericit. Însă omul ajunge să fie jefuit de amintiri, într-un act de refuz vehement de conștientizare a trecutului ca trecut. \textit{Jacking-in}-ul este un mecanism de apărare pentru personajele concepute de Bigelow, nu doar pentru Lenny (Palermo, 2020). Protagonistul este clișeul dealer-ului care se folosește de propria sa marfă. O portretizare similară se regăsește în \textit{Videodrome} (1983, r. David Cronenberg), însă \dedouble dependența de imagini\sqtworight este mult mai atașată de limbajul fantezist, horror, al filmului. Protagonistul lui Cronenberg practic și-o caută cu lumânarea; Lenny este mai degrabă într-o stare constantă de derivă, încercând pe cât posibil să-și mențină echilibrul moral. \textit{Videodrome} se simte ca o halucinație, ca o experiență mai degrabă apropiată de una divină, pe când \textit{Strange Days} te pune în fața faptului împlinit.\par
	
	Deoarece SQUID permite accesarea amintirilor prin conexiune directă la cortexul prefrontal, acesta presupune o lipsă a interpretării. Simțurile utilizatorului sunt asaltate nemilos, iar omul ce poartă dispozitivul este vulnerabil, incapabil să influențeze întâmplările ce-i sunt prezentate. Cât timp utilizatorul este jacked-in, este în agonie și oricând este posibil să moară din cauza suprastimulării. De aceea a fost și ilegalizat, însă asta nu oprește pe nimeni din a încerca dispozitivul oricum. Intenția moralizatoare a lui Bigelow este clară aici: când Lenny devine „martor” la cele mai inumane momente capturate, încearcă să iasă pe cât posibil. Atât el, cât și noi, din public, simțim aceeași durere, același sentiment visceral pe care nu am fi vrut să-l lăsăm să ne acapareze. În cuvintele ei:\par
	
	\begin{displaycquote}{Willistein1995a}
		It is meant to be awful. The shower scene in \textit{Psycho} \textemdash that dramatic fulcrum \textemdash it had to be extremely intense within the thriller construct.
	\end{displaycquote}
	
	Lenny nu-și dă seama că oricât ar iubi-o pe Faith în continuare, o transformă într-un obiect prin faptul că se folosește de înregistrări pentru propriul său folos. Trăind în propriul trecut, crede că poate acesta încă se regăsește și-n prezentul exterior SQUID.\par
	
	Amintirile nu-și pot regăsi aceeași ființare ca realitate. Citându-l pe Marcel Proust, Roland Barthes descrie această experiență mai bine decât aș putea eu:
	
	\begin{displaycquote}{BarthesCameraluminoasa2009a}[p.61-62]
			Într-o seară de noiembrie, la puţină vreme după moartea mamei mele, puneam în ordine nişte fotografii. Nu speram să o "regăsesc", nu aşteptam nimic de la \guillemetleft aceste fotografii ale unei fiinţe, în faţa cărora \textbf{ţi-o aminteşti mai puţin decît mulţumindu-te să te gîndeşti la ea}\guillemetright  (Proust, 111, 886) […] Istoria este isteri­că: ea nu se constituie decît dacă este privită - iar \textbf{pentru a o privi, trebuie să fii exclus din ea.}
	\end{displaycquote}
	
	Prin contrast cu SQUID, fotografia ți se prezintă ca o reprezentare voluntară a realului, iar obiectul real nu-ți va revela niciodată esența lui \cite{BarthesCameraluminoasa2009a}. Fotografia îți permite să te distanțezi la nevoie de obiectul reprezentat \cite{secSontagPlatosCavea}, să lași amintirile să se piardă înapoi în negura istoriei. Mace (Angela Bassett), unul dintre singurele personaje care refuză experiența SQUID, încearcă să-l facă și pe Lenny să înțeleagă aceeași idee:
	
	\begin{displayquote}
		This is your life. Right here, right now! It’s real time, you hear me? [...] Time to get real, not playback. [...] These are used emotions; time to trade them in. Memories were meant to fade, Lenny. They’re designed that way for a reason.
	\end{displayquote}
	
	Scoptofilia presupune că Lenny vede încă înregistrări cu Faith și chestia asta este bine cunoscută de practic toată lumea din viața lui. Nu există o indicație mai clară că acesta nu ar consuma filmări cu conținut pornografic. Societatea portretizată este deja una care este obișnuită cu a fi privit, începând chiar cu protagoniștii: Faith știe că a fi privită este singurul mod prin care-și poate asigura supraviețuirea, Max este plătit să urmărească oameni pentru a le asigura „siguranța”, Lenny trăiește de pe urma urmării.\par
	
	Mace este specială deoarece este singura care nu cade pradă acestei utilizări nesănătoase a privirii obsesive.\par	
	\printbibliography
	
\end{document}